\addcontentsline{toc}{chapter}{Abstract}

\begin{abstract}
	Smartphone overuse and internet addiction is strongly correlated with a sharp rise in mental health problems and reduced productivity. Social media and gaming applications (``apps'') earn money off of users' screen time and thus leverage techniques from behavioural science to cause users to form unconscious habits which lead to increased usage. This greatly detracts from an individual's productivity and ability to focus on difficult tasks.

	Many attempts have been made to develop technologies which help individuals reduce these effects in the form of well-being apps. The most effective well-being apps aim to use the same techniques from behavioural science used to hook users to reverse their own effects. This project seeks to continue the work of one such well-being app, closing the gap in product quality and feature offerings by leveraging recent advancements in application development methods.

	The new app will be developed in collaboration with a consultant neuroscientist, and will be a novel effort in exploring the translation of behavioural ``hooks'' into a set of software engineering requirements that can be implemented in an app. The resulting work is a highly interdisciplinary blend of psychology, neuroscience, software engineering and user experience design.

	On top of enhancing the original product and its applications in behavioural rehabilitation with new features, the resulting work facilitates developer on-boarding, reduces development time and implements analytical methods for developers to quantitatively assess how effective new features are at helping users improve their focus and productivity.
\end{abstract}