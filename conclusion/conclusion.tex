\chapter{Conclusion}

\label{ch:conclusions}

\section{Summary of Work Done}
A large amount of literature has been reviewed in pursuit of understanding the mechanics of human attention and addictive app design. A plan for implementation has been laid forth, with a flexible set of requirements catering to the nonlinear progression of solo-software development. A undocumented minimum working example has been developed which re-invents the user interface of the existing app. At present, the MWE supports the task list and focus timer functionalities only, with notification blocking on the way.

% Clarify MVP and MWE 
% Clarify IAD with the DSM

\section{Applications}
The application, if successful, could see use as a tool in rehabilitation of individuals suffering from chronic Internet Addictive Disorders. Corporations and individuals are seeking ways to reduce distraction from productivity, and so this app could see a commercial or retail flavours to cater to those audiences.

\section{Future Work}
The next steps in this project would be to start with user feedback cycles. At present, we do not expect enough users to be able to implement A/B testing. However, feedback will be built into the app and aggregated for weekly design reviews. The present goal is to have an MVP that implements goal setting, to-do lists and notification blocking up by the beginning of the Summer term, so that we might start collecting usage analytics from students who are studying for exams.