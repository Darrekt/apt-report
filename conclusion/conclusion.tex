\chapter{Conclusion}
\label{ch:conclusions}

\section{Summary of Work Done}
A large amount of literature has been reviewed in pursuit of understanding the mechanics of human attention and addictive app design. A plan for implementation was laid forth and accomplished, with a flexible set of future requirements catering to the nonlinear progression of solo-software development. A minimal app has been developed and positive reviews from initial testers shows promise in the efficacy of attention-preserving features. The app currently offers a fully cloud-synced structured system for planning and breaking down tasks, with full notification-blocking functionality.

\section{Applications}
Internet Addictive Disorders (as generally known by the public) as a term lacks standardisation and is not yet recognised as a disorder in the Diagnostic and Statistical Manual of Mental Disorders (DSM-5). The DSM-5 does, however include Internet Gaming Disorder. The application, if successful, could therefore see use as a tool in rehabilitation of individuals suffering from chronic Internet Gaming Disorders, while simultaneously bringing awareness to the more comprehensive set of problem-causing technologies whose impacts may one day be formalised as Internet Addictive Disorders.

Corporations and individuals are increasingly seeking ways to reduce distraction from productivity, and so this app could see commercial or retail flavours to cater to those audiences. Once a good privacy policy is in place, informed users can consent to contributing usage analytics which may be helpful for ongoing research in neuroscience and human-computer interaction related to Internet Addictive Disorders.

\section{Future work}
Due to many unforeseen deployment issues, much of the effort in this run of the project was dedicated to building a codebase for legacy. Much of the tedium and complexity of building a production-quality app has been automated as part of the project's infrastructure. As such, there is room to add many features on to the existing app, without requiring more than a junior developer's level of expertise.

\paragraph{Creating a privacy policy and releasing the app on public stores.} Since one of the ways the app can help the user is by logging usage data, a detailed privacy policy is necessary to provide full transparency to users about how their data is used. While no malice is intended on our end, the privacy policy is also required for app store publication, and is reviewed in line with the sensitive permissions that the app asks users to grant it on the host operating system. This review process typically takes 2-3 weeks, and at the time of writing was subject to more delays, which made app store publication not possible within the scope of the project.

Additionally, the review process for app store publishing requires numerous assets and processes to be in place, which are disjoint with the original engineering-based scope of this project. As such, publishing the app requires the involvement of a graphic designer, as well as considerable effort in creating branding and marketing assets. Furthermore, creation of developer accounts requires a monetary investment which should lie with a central, consistent entity that is owned by the project, and not the people working on it. \$125 USD is required for developer licenses, and additional costs may be incurred in cloud-computing provider costs, domain ownership and custom email accounts for customer support.

\paragraph{Migrating to an SQL-based schema.} Firebase's Firestore is an opinionated database which only supports a noSQL schema, which favours read-heavy applications. Within is a write-heavy application, and should the user base scale to a reasonably large size (for example, 100,000 daily active users), the cost incurred in using Firebase may be higher than if the app were implemented on a separate SQL database. However, the implementation of a distributed SQL database that supports websocket listeners for real-time synchronisation is non-trivial and would occupy the scope of a full project of it's own, and thus is recommended as a future extension if the app starts to gain a large, active user base.

\paragraph{Ongoing development.} The app currently has a small pool of internal testers who may stop using the app for any number of reasons. It is important to constantly increase the pool of test users to continue iterative development and root out bugs in the application.

\paragraph{Adding more blocking functionalities: web-extensions and more. }
A web-client was originally proposed as as stretch goal for this project. However, after considering multiple rounds of user feedback, a few users mentioned that they enjoyed site and app-blocking functionalities from other apps such as BlockSite and Hold. Additionally, the interface for Within is rather minimal and does not require the space of a full browser window. It thus makes sense to implement Within as a browser extension instead of a web client, which will permit for the replication of site-blocking functionality according to a user's specified blacklist.

\href{https://blocksite.co/}{BlockSite} additionally attempts to add app-blocking functionality to its Android app, which displays a pop-up over blacklisted apps should the user try to access them. Naturally, there are limitations to this; the user can simply click off the pop-up and continue using the blacklisted app. Nonetheless, adding an additional layer of deterrence is useful to strengthen the user's resistance to their own involuntary impulses, and is greatly in line with the principles of neuroscience that will make this app an effective productivity-booster.