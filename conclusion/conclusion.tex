\chapter{Conclusion}
\label{ch:conclusions}

\section{Summary of Work Done}
A large amount of literature has been reviewed in pursuit of understanding the mechanics of human attention and addictive app design. A plan for implementation has been laid forth, with a flexible set of requirements catering to the nonlinear progression of solo-software development. A undocumented minimum working example (MWE) has been developed which re-invents the user interface of the existing app. At present, the MWE supports the task list and focus timer functionalities only, with notification blocking on the way.

\section{Applications}
Internet Addictive Disorders (as generally known by the public) as a term lacks standardisation and is not yet recognised as a disorder in the Diagnostic and Statistical Manual of Mental Disorders (DSM-5). The DSM-5 does, however include Internet Gaming Disorder.

The application, if successful, could therefore see use as a tool in rehabilitation of individuals suffering from chronic Internet Gaming Disorders, while simultaneously bringing awareness to the more comprehensive set of problem-causing technologies whose impacts may one day be formalised as Internet Addictive Disorders. Corporations and individuals are seeking ways to reduce distraction from productivity, and so this app could see commercial or retail flavours to cater to those audiences.

As long as users are notified and consent to it, the app's usage data and analytics may be useful for ongoing research in neuroscience and human-computer interaction related to Internet Addictive Disorders.

\section{Future Work}
The next steps in this project would be to start with user feedback cycles. At present, we do not expect enough users to be able to implement A/B testing. However, feedback will be built into the app and aggregated for weekly design reviews. The present goal is to have an minimum viable product (MVP) that implements goal setting, to-do lists and notification blocking up by the beginning of the Summer term. This MVP can then be used as a alpha-release of the app so that we might start collecting usage analytics and feedback from students who are studying for exams.