\chapter{Introduction}

\section{Motivation}
One does not need to quote scientific resources to know that the smartphone has been one of the most beneficial additions to modern society. The smartphone and the rich array of features it brings has changed the way many industries operate and increased speed of communication and efficiency. Banks and utility companies no longer need to waste resources sending letters, colleagues can have video conferences at the drop of a hat and friends and family are just a message away.

However, some of the features that bring us such joy are the very same features shown to have a very detrimental effect in some users. Social media, emails and games are a constant distraction that exist on the very same device that is used for work and productivity, be it a smartphone or computer. There is much compelling scientific literature suggesting that the overuse of smartphones and the effects of its addictive applications is highly correlated to increased levels of sleep disorders, depression, anxiety and lower life satisfaction \cite{abi2020smartphones, lee2014hooked, demirci2015relationship}.

In this report a particular focus is given to the negative effects it has on the individual's attention capacity, motivation and focus. While some references will be made to extreme cases of highly-addicted audiences, the general demographic under study is the average adult, ranging from college students to working professionals.

\subsection{Problem Statement}
% What's the problem? Lay it out in a compelling and motivational way
The presence of a smartphone or otherwise connected device has shown to have adverse effects on one's productivity and ability to stay focused \cite{thornton2014mere}. This is especially so for individuals whose work involves many hours at at desk, working closely with a connected device, who are unable to decouple their personal devices from doubling-up as a productivity and communication tool for work.

This digital dependence requires the individual to always exercise some level of self-restraint, to prevent distraction from the priority task at hand, and its existence not at all coincidental. This report will show that the most successful modern smartphone applications are addictive by design and have very successfully secured a place for themselves in their users' lifestyles. The user is therefore, without external education and intervention, quite effectively helpless to identify and remedy the negative effects of smartphone use that have crept into their lifestyle. In extreme cases, this may result in what is generally known as an Internet Addictive Disorder (IAD). 

\subsection{User Attention as a Commodity}
% User attention as an industry and key business focus
Social media and gaming applications (henceforth abbreviated as "apps") generate profits by retaining the attention of a user. Social media apps might monetize this by generating advertising revenue off of a user's screen time, or offer the user additional premium features requiring an in-app transaction to unlock.

Gaming apps have evolved to follow this model of monetizing continued user engagement: games in the early 2000's delivered a full product on an upfront purchase, seeking to deliver an experience that would encourage users to buy a sequel, but modern games now attempt to monetize continued user playtime by offering "microtransactions" in the form of in-game cosmetics or bonuses on top of the base game purchase.

As of June 2020, of the 5 companies to have ever been valued over a trillion dollars, 4 of them are technology companies that profit off of user engagement with their products: Apple (\$1568b), Microsoft(\$1505b), Amazon(\$1337b) and Google(\$953b), leaving the winning business models of the decade before such as Walmart(\$337b) and Johnson \& Johnson(\$366b) in the dust \cite{pwctop100}. Even Facebook(\$629b), whose business revenues are almost purely from monetizing user attention, easily doubles the value of the winning business models of the past.

Twitter (a social media company whose revenue comes primarily from advertising) even goes so far as toting the metric of "monetizable daily active users" (mDAU) in investor press conferences \cite{twtrTranscript}, referring to users which the company may show ads to. Investors of such business models focus so heavily on such metrics that, despite beating traditional financial health indicators (such as earnings estimates) by a sound margin in the third fiscal quarter of 2020, Twitter lost 20\% of its stock value in a single day simply on the announcement that they had only gained 1 million new users instead of the projected 9 million \cite{twtrLoss} \cite{twtrTranscript}.

Therefore, the new hot commodity of the 21st century is no longer gold, silver or oil, but user attention. The attention, engagement and growth of an app's user base is so lucrative for these companies that they have completely changed how their business models are evaluated and priced.

\subsection{Addictive App Design}
In her book "Addiction by Design", anthropologist Natasha Dow Schüll illustrates the concept gamblers who are in the "Machine Zone", a tracelike state which is an almost perfect state of escapism \cite{schull2014addiction}. Players in this zone are no longer playing for the original thrill of winning, but to simply extend their time in that state. Worse still is the fact that this state may be induced with careful design of the user experience. The optimal state of profitability for a company is if they can achieve this in their users, who will then seek the user experience offered by their app voluntarily instead of the app having to find ways to pull the user's attention.

Social media and gaming apps are therefore carefully designed in line with key concepts from behavioural science to be highly addictive, aiming to capture the user's attention and keep them coming back to interact with the application. This is loosely referred to in the industry as a user's level of "engagement" within the application. To increase and retain user engagement, techniques best described as "addictive software design" \cite{neyman2017survey} are employed, including but not limited to:

\begin{itemize}
    \item Variable rewards, as in gambling slot machines, maintaining an element of unpredictable positive reinforcement that keeps users coming back for more \cite{neyman2017survey}.
    \item ``Bottomless" content with infinitely scrolling lists, experimentally proven to increase consumption \cite{neyman2017survey}.
    \item Ease of use and consistent user-interface elements across apps, allowing users to conveniently feel a sense of familarity when accessing their apps across all platforms (PC, smartphone, tablets and wearables).
\end{itemize}

Such mechanisms are deliberately crafted to appeal to the primitive parts of the brain responsible for unconscious habitual compliance (the limbic system), rather than those associated with rational, forward and deliberate thought (the prefrontal cortex). This enables the tendency to act on habit and impulse rather than exercise conscious thought. We will see later in greater detail just how these apps exploit and reinforce this tendency to a point which can be extremely destructive.

% Willpower depletion and device abstinence even among aware users
With the onset of COVID-19 and the rise of the remote workplace, the line between work and play has grown even more vague. Individuals report higher screen time and video game participation which has been correlated to decreased mental health \cite{colley2020exercise}. Smartphones and computers are key pieces of technology that are essential for connectivity and productivity in the age of remote work. The devices themselves do not pose a problem, but the access they give users to the attention-grabbing applications they contain have significant effects on the individual's ability to remain focused at work. The exercise of willpower alone, while possibly yielding some results in the short term, is not the solution to the problem.

\subsection{Present Solutions}
% What has been done?
Therefore, many efforts have been made in the fields of neuroscience and psychology to raise awareness of the dangers of digital addiction. Similar addiction-based problems such as gambling and drug use may employ a ``cold-turkey" approach, but depriving a modern worker of a smartphone and computer almost completely robs them of function in today's digital age. One approach to remedy device overuse is, paradoxically, the use of so-called well-being apps, which attempt to address problems caused by internet addiction and overuse.

There are many more well-explored examples of insidious software design and behavioural manipulation to exacerbate internet overuse \cite{sagePaper2020}, but the arguments proposed so far sufficiently motivate the creation of a solution that does not require the complete abstinence of smart-device usage.

Many examples of so-called well-being apps have been made, in a paradoxical attempt to design apps that would enhance users' productivity or state of mental health. However, \cite{sagePaper2020} highlights that the such apps do not employ the same behavioural "hooks" that social media and gaming apps use. They often yield a short period of initial use, propelled by motivation, and then fall out of favour of their users within a few months.

Therefore, the work previously done in \cite{sagePaper2020} proposes the concept of attention-preserving technology, which leverages the same behavioural science and software engineering techniques from addictive apps to help users stop using them and reverse their detrimental effects. The authors also produced a proof-of-concept: a well-being app called Sage. In summary, their work uses the development of Sage to highlight the failings of the well-being apps that came before it, proposing a set of design principles for future effective attention-preserving app design.

The Sage application helps users regain their focus and productivity through task and time-tracking features. However, given the short development time, it currently exists only as an Android application with limited functionality. Despite this, an initial round of evaluation on the alpha release yielded positive reviews from surveyed testers, claiming that it improved their awareness of their own productivity and that they would continue using the application.

\section{Objectives}
% What happens after? Why is it interesting?
The work documented in this report thus seeks to extend Sage with the following objectives:

\begin{itemize}
    \item Engineering objective: Re-implementing the code base with methods that would be more suitable for lean development teams to continue their work, as well as adding infrastructure for quantifiable measurement of how well the application achieves its objective.
    \item Paradoxical social objective: Since Sage did not get a lot of development time and only implemented basic features, this project seeks to further explore how to further leverage the technologies and methods that cause the problem of digital addiction to reverse the problem.
    \item Academic / Exploratory objective: Contrary to Sage, this run of the project involves the cooperation of a neuroscience subject-matter expert through the development process, which will highly increase the interdisciplinarity of the resulting work and allow us to explore the relation between neuroscience and software engineering in addictive app design.
    \item Application objective: Creating a product that may prove useful in prevention and rehabilitation of attention deterioration as a result of device overuse. This could even see use as a commercial product for companies wanting to help their employees improve their focus at work.
\end{itemize}

In summary, this run of the project aims to improve the current implementation of Sage. The application will be made available to a much larger user base by re-implementing the code base using cross-platform development frameworks. The user interface is then to be given a complete overhaul to improve the smoothness of the user experience and increase user engagement. More features will be proposed, implemented and developed in collaboration with a consultant neuroscientist, in an attempt to deliver higher product efficacy through interdisciplinarity. Finally, an attempt will be made to create and implement quantitative evaluation methodologies that allow the product team to identify which features are most effective at helping users improve productivity and reduce screen time.