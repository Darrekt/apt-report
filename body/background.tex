\chapter{Background}

\label{ch:background}

\section{Iterative Software Development}

\section{User Experience Design}
% User Journeys

% Jank, disruptions, etc

\section{Cross-platform Development}



\section{The Existing Sage Application}
% Intro
\subsection{Functional overview}
After granting the app permissions for notification access in an onboarding sequence, the app starts up and offers its core functionality: a task list. Users may create and edit tasks. Each task may be recurring daily, and may block notifications. Users may then at any time start working on a task, triggering a running timer which may block all incoming notifications while active. A task can be completed, or returned to any time later with a break. All this activity is stored within the client and can be displayed to the user through graphs as a form of feedback on their productivity.


\subsection{Limiting issues}
However, there are several issues and limiting features with the current implementation of Sage.

\begin{itemize}
    \item Does not fully block notifications. Certain edge cases allow notifications to still slip through.
    \item Currently implemented as an Android app, no extensibility to iOS or desktop users.
    \item Completely local: no data backups or export, no online cloud synchronisation.
\end{itemize}


\section{Project (Re-)specification}
% Feasibility study
Proposing a re-implementation software project naturally begs the following questions:
- What advantages will the effort yield, and is it worth the effort?
- Will it perform as before, or are there be any compromises as a result of the move?

As such, some initial research had to be conducted to address the concerns. 
Studying the existing codebase and Android developer documentation 

% Requirements Gathering
As in any other software engineering problem, the first step is to gather an initial list of requirements that will form the specification of what the final product must achieve. 



The re-implementation of the Sage application thus should fulfill the following set of requirements in its MVP:

\underline{Functional Requirements}
\begin{itemize}
    \item Re-implement the full feature set of the existing application.
    \item Fix notification blocking functionality with the use of \textit{Do not Disturb} mode, instead of the current implementation.
    \item Add online cloud storage and synchronisation of user data.
\end{itemize}

\underline{Non-functional Requirements}
\begin{itemize}
    \item Must be performant without excessive jank, jitter or loading delay.
    \item In the event of a crash, the application should recover gracefully and inform the user of any inconsistencies in their data caused by the crash (interrupting a focus period, etc).
\end{itemize}


\section{Implementation Plan}


\section{Evaluation Plan}


\section{Ethical, Legal and Safety Plan}