\chapter{Background}
\label{ch:background}
With the goals and objectives of the project clearly laid out, it is now necessary to explore the factors which drive digital dependence. This chapter explores the concepts in biology and neuroscience which cause "hooked" users to become unfocused and unproductive despite their best efforts to do otherwise. It will then highlight a few clear examples of how the design of modern apps leverage these concepts and then offer a design and implementation plan that

\section{Biology Primer}
To develop an app to fix the problem of attention deficiency, a short exploration into the mechanics of attention preservation is necessary.
Within this scope, our focus will be on two different parts of the brain: the limbic system and the prefrontal cortex.

\subsection{The Limbic System}
The Limbic System carries out many functions that are considered "primal urges". Emotional responses to food, smell, social cognition, emotional memory, sexual behaviour and most importantly for this context, addiction and motivation \cite{rajmohan2007limbic}. All these functions can be summarised and grouped by one common concern: self and species preservation.

The limbic system (henceforth referred to as the "chimp"), is adept at remembering emotional responses to previously experienced situations and inducing the optimal response for self-preservation, should it find itself in a similar one. These responses can be associated with negative emotions, such as the fear one experiences when spotting a predator, or positive, such as the satisfaction and drive upon finding a food source. This is a gross over-simplification of the capabilities of the limbic system, but the message is clear: it is the part of the brain responsible for inducing impulses that it deems will result in the highest chances of survival and comfort.

However, with respect to self-control and achieving one's long term goals, this may be undesirable. The stress hormone (cortisol) produced when working under tight deadlines with complex problems is not unlike that produced in life-or-death situations in the wild. The chimp has no way of differentiating this and therefore the stress response induced by work is often registered as an existential threat, which encourages impulses of procrastination or indulging in escapism as part of a "flight" response.

\subsection{The Prefrontal Cortex}
The Prefrontal Cortex (PFC) is the part of the brain largely associated with conscious thought and rational self-control. An extremely well-cited study by Miller and Cohen concluded that in general, actions that require critical thought against our base instincts always involve the engagement of the PFC, and that test subjects with PFC impairment performed poorer on such tasks \cite{miller2001integrative}. One such example was the sorting of cards by the colour verbally printed on them, even when the card itself was another colour (e.g. a blue card with the word "red" printed on it would be classified as red).

One prevailing theory is therefore that the PFC and the limbic system are often at odds in tasks involving self-regulation \cite{heatherton2011cognitive} and that successful self-control is a balancing equation of the strength of so-called "bottom-up" impulses and "top-down" conscious thought. This sets the stage for our narrative: that the "chimp" within us and the "human" are always at odds when it comes to work and procrastination. While the PFC may wish to act in favour of our long-term goals, it often loses to the "chimp" and we are left with only a running commentary on our actions, despite our best wishes to do otherwise.

\section{Self-regulatory Failure}
Resisting the urge to check one's phone or play video games is a prime modern example of self-regulatory failure that reflects much in common with addiction-like behaviour in dieters, smokers and substance abusers. App-fueled procrastination may not be as lethal as substance abuse or obesity, but there nonetheless exists a very slippery slope into internet addiction that may ensnare the average person. Heatherton and Wagner elaborate in \cite{heatherton2011cognitive} that there are a few common causes for self-regulatory failure, which can be separated into two categories. "Top-down" factors are associated with reduced capacity or strength of conscious self-control such as negative moods and self-regulatory resource depletion. "Bottom-up" methods refer to unconscious or natural impulses such as cue exposure or lapse-activated consumption.

\subsection{Resource Depletion \& the Strength Model of Self-Control}
Baumeister and Heatherton \cite{baumeister1996self} initially proposed in 1996 that conscious self-control against one's impulses draws from a global resource pool that depletes as one uses it further. Since then, many studies have released results in support of this hypothesis. The strength model by construction dictates that it is possible to exhaust one's self-control and must be replenished like any other resource.

This suggests that as long as a distracting alternate stimulus is present, self-regulatory failure is not a question of "if", but "when". For traditional office work settings, this effect is ameliorated by the social norms enforced by the office culture. However, for a person working from home, the distraction unfortunately exists on the same device that enables productivity and there is no one else around to keep the chimp in check.

\subsection{Cue Exposure}
A cue in this context is defined as a piece of sensory stimulus which has an association to a certain consumption-related behaviour. The smell of food for a dieter, the sight of a beer for a heavy drinker. Cues have been shown to increase cravings, draw attention and increase likelihood of consumption \cite{jansen1998learning}. Furthermore, multiple studies have shown that individuals are often unaware of how cues affect them on a conscious level\cite{stacy2010implicit}, and so it becomes difficult for a struggling individual to pinpoint what it is exactly that causes their self-control to lapse.

The first example of addictive software engineering is the push-notification, one of the core tools of user-experience (UX) design. The premise seems simple: software engineers needed a way of bringing a user's attention to key information. This is not a "new" phenomenon, as pagers and message alerts have been around for as long as phones have existed. However, the graphical interface of the smartphone takes this to a new level. The modern push notification engages many senses: a kinetic vibration, an audio alert and a visual alert window.

Both Android and iOS have robust application programming interfaces (APIs) that enable developers to exercise a very high level of control over how their app sends and handles notifications \cite{androidnotification}. In the example of a social media message from a friend, this allows for extremely alluring design: an avatar of the friend and a short preview of the incoming message. Users are even empowered to make the experience of receiving a notification as enjoyable for themselves as possible. Facebook Messenger allows users to set nicknames for their friends. Most phones allow users to set specific alert sounds on a per-contact basis, allowing a user to know from audio perception alone who they are receiving a message from. While this feature feels like one that helps the user "filter out the noise", one might argue that it in fact facilitates distraction, by allowing the user to have a stronger, more focused response to a self-set cue.

Cue exposure need not even be as explicit as a push notification. The experience of using a social media app or playing a game is extremely pleasurable. Beautiful colour palettes, high-resolution artwork and buttery-smooth animations are a staple in successful apps. All these factors enable the "chimp" to register a positive emotional response to act of phone use. Therefore, the very presence of the phone is a form of cue exposure that constantly forces the user to exercise self-restraint.

\subsection{Lapse-activated Consumption}
A 1975 study showed that the consumption of a small amount of an addictive substance (in this case, a milkshake on a test audience of dieters) paradoxically caused dieters to consume more food afterwards, in contrast to the control group of non-dieters who ate less \cite{herman1975restrained}. The exact reason as to why this phenomenon occurs is not fully clear, but the implications of its existence are. Picking up one's phone for a short break from work can often snowball into a longer-than-intended session of scrolling social media. Playing "just one round" of a game often ends up being anything but. We will explore in later sections what mechanisms are in place that potentially cause such activities to be so difficult to quit.

\subsection{Consequences of Distraction}
Levy and colleagues showed that not only is the occurence of a disturbance a problem, but the "richness" of that disturbance was also detrimental to subjects' performance on cognitive tasks \cite{levy2016effect}.

\subsection{Motivation and Persistence}
We have seen so far that the outcome of a self-restraint task comes down to whether the strength of bottom-up impulses exceeds that of our top-down intentions. Most of the time, this tends to be the case. Having explored all the ways in which self-control can fail due to how strong our in-built tendencies are, perhaps then there is a way in which we may use these natural tendencies to our advantage, or at least increase the strength of top-down control.

\subsection{Strengthening Self-control}
% How can one increase their desire and motivation to work on long-term goals?
Muraven, Baumeister and colleagues conducted a study in which subjects performed consistent amounts of self-enforced exercise over two weeks. While this caused slightly decreased self-control capability on the day, this ``strength drain" diminished rapidly by the day and resulted in increased success in other completely unrelated self-restraint tasks \cite{muraven1999longitudinal}. Muraven took these results again in a later study and found that "smokers who squeezed a handgrip or avoided sweets for two weeks before quitting cigarettes remained abstinent longer and had fewer lapses overall as compared to smokers who practiced tasks that did not require self-control" \cite{muraven2010practicing}. 

% Is there any way to convince the chimp to want to do work?
These studies, and many others like it, strongly support the previously mentioned strength model of self-control. Metaphorically, self-control behaves very much like a muscle: its capabilities are slight diminished following immediate use and while it is possible to fatigue it significantly, it is also possible to increase its capabilities with consistent amounts of progressive load.

\subsection{Impairing Impulses}
Cinciripini and colleagues investigated the effects of schedules and gradual consumption reduction in smoking cessation. Participants were all given uniform education on the psychology of nicotine addiction and methods of coping \cite{cinciripini1995effects}. Control over consumption was exercised in two forms: scheduling, where participants were allowed to smoke at designated and progressively lengthened intervals, and reduction, where consumption was reduced by a third of the subject's baseline consumption each week. The scheduled and reduced group unsurprisingly performed best, with the greatest reduction in mean cigarette consumption and reduced frequency of urges and severity of withdrawal symptoms. Reduction alone contributed most to the reduction of urge frequency, and scheduling seemed to have an effect on mean consumption. 

Results from studies such as these suggest that the controlled appeasement of urges may be a key factor in helping to diminish the effects of bottom-up impulses and serve as a basis for productivity schedules such as the pomodoro technique. 

\section{Addictive Design}
The principles of behavioural science examined so far dictate that a successful well-being app therefore has to not only help the user form healthy habits that help them avoid self-regulation pitfalls, but offer new "hooks" to help to keep their inner chimp from becoming overly restless.

\subsection{Social Media}
Social media offers novelty at every turn, with variable rewards

\section{Gaming}
Gaming, on the other hand, are capable of offering a dense, fast-paced, highly engaging experience.


\section{The Existing Sage Application}
\subsection{Functional overview}
After granting the app permissions for notification access in an onboarding sequence, the app starts up and offers its core functionality: a task list. Users may create and edit tasks. Each task may be recurring daily, and may block notifications. Users may then at any time start working on a task, triggering a running timer which may block all incoming notifications while active. A task can be completed, or returned to any time later with a break. All this activity is stored within the client and can be displayed to the user through graphs as a form of feedback on their productivity.


\subsection{Limiting issues}
While the work done so far is an admirable starting effort, there are many potential points of improvement from a user-experience perspective. There are several issues and limiting features with the current implementation of Sage, from both a UX design and software engineering perspective:

\underline{Design \& User Experience}
\begin{itemize}
    \item The selected colour palette is rather dull, with no possible customisation on behalf of the user.
    \item All the assets and images are static, with no animation or movement to drive user engagement.
    \item Sizing of interactive elements is not well-standardised, and in most cases are oversized.
\end{itemize}

\underline{Software Engineering}
\begin{itemize}
    \item The notification blocking feature is faulty. Certain edge cases allow notifications to still slip through.
    \item Currently implemented only as an Android app, no extensibility to iOS, desktop or mobile web users.
    \item Completely local: no data backups or export, no online cloud synchronisation.
\end{itemize}


\section{Project (Re-)specification}
Modern individuals are unable to focus and stay productive for a multitude of reasons. This project seeks to solve that problem by creating a user experience that leverages the concepts explored so far, such that less deliberate exercise of self-discipline is necessary for the user to remain on-task.

% Feasibility study
Proposing a re-implementation software project naturally begs the following questions:
- What advantages will the effort yield, and is it worth the effort?
- Will it perform as before, or are there be any compromises as a result of the move?

As such, some initial research had to be conducted to address the concerns.
Studying the existing codebase and Android developer documentation

% Requirements Gathering
As in any other software engineering problem, the first step is to gather an initial list of requirements that will form the specification of what the final product must achieve.

The re-implementation of the Sage application thus should fulfill the following set of requirements in its MVP:

\underline{Functional Requirements}
\begin{itemize}
    \item Re-implement the full feature set of the existing application.
    \item Fix notification blocking functionality with the use of \textit{Do not Disturb} mode, instead of the current implementation.
    \item Add online cloud storage and synchronisation of user data.
\end{itemize}

\underline{Non-functional Requirements}
\begin{itemize}
    \item Must be performant without excessive jank, jitter or loading delay.
    \item In the event of a crash, the application should recover gracefully and inform the user of any inconsistencies in their data caused by the crash (interrupting a focus period, etc).
\end{itemize}


\section{Implementation Plan}
% What is agile software development
\subsection{Software Development Practices}
An excellent starting point for planning this project is to follow the principles laid out in agile software development. Agility is a concept often used to in business strategy and operation, referring to an organisation's ability to adapt to quickly changing business demands and put themselves in the most competitive position possible. This concept has naturally spread to software development practices and there now exist frameworks for teams to follow such as Scrum, lean software development and Kanban.

% TODO: Picture


\subsection{Time Management}
% Motivation for developing features iteratively
There are 3 key members in this project team in this interim phase: the part-time solo developer, the supervisor and a consultant who doubles as the client. Given the timing and nature of term-time work, the structure of development cycles must be flexible to accomodate for unforeseen workload variance on behalf of the developer or consultant. Core members are only available on a part-time basis until the summer term,

We therefore structure work into variable 2 or 1-week sprints, taking after agile software development practices instead of a typical waterfall Gantt chart structure. 

Each 

\subsection{Plan of Deliverables}
% Rough timeline, release plan and feature map


\section{Evaluation Plan}
% Proposed evaluation metrics
Instead of the typical academic approach of conducting individual user studies on a select sample, this app hopes to build feedback in to the app as part of the user experience. The app is thus hosted on a cloud provided with a generous "free tier" allowance, allowing us to experiment with a sizeable user base of what we estimate to be up to 500 users before cloud computing costs actually start to kick in.

For this study, the concerns are not to develop the world's best-selling productivity app or to gain traction with a massive user base. This preliminary phase of the project is to find a way to measurably determine how effectively the Sage platform does its job of helping users stay on task. The refinement, re-selection and validity of these metrics will have to be determined as part of future work when the app is released in beta to the test audience. As an example, we may wish to measure a given user's productivity on a day-to-day basis by capturing a "productive window" in which a user continues to execute tasks without a significant (15 minute) disruption.

The app will be released to students in Imperial College London through the Android and iOS distribution channels, with its intentions clearly stated in both publicity efforts and as part of the in-app onboarding process. We hope to complete development on the alpha release in time for the Easter holiday, when students will be focusing on studying for exams.

\section{Ethical, Legal and Safety Plan}
This project has no actionable safety concerns, since there is no physical product in which the user might harm themselves with. The primary concerns are with the use of third-party intellectual property and the safe and compliant use of user data for analytics and product improvement.

\subsection{Software Licensing}
This project will leverage the use of open-source software libraries and frameworks, the largest example of which is React Native. Open-source, by definition, allows the unrestricted redistribution and modification of a given piece of software for any purpose. As a secondary precaution, all packages used will be derived from the Node Package Manager (NPM) registry, which effectively serves as a central distribution point of open source packages pertaining to the javascript ecosystem (in which this project resides).

Most, if not all packages registered on NPM have an open-source approved license selected for their project. That is, a license which is in accordance with the open-source philosophy of freely useable, modifiable and shareable code. The list can be found at \href{https://opensource.org/licenses}{the open source initiative organisational website}. Should the exceptional case arise in which a package is used which does not use one of these licenses, it will be explicitly justified and disclosed in its implementation documentation. However, this should technically never arise.

\subsection{User Data Usage and Compliance}
This study will use the General Data Protection Regulation (GDPR) as a basis of rules to follow in the ethical collection and use of user data. The following subsections will outline the parts of the GDPR deemed relevant to the nature of data used in this study, and perform a risk assessment of where grey areas may occur.

In the interest of brevity, the GDPR checklist can be found \href{https://gdpr.eu/checklist/}{here}. In addition to having completed this checklist, the Sage application will adopt the following policies for the collection of user data:

\begin{itemize}
    \item Users will be explicitly informed during the onboarding process what data we collect for evaluation metrics and how we will use it. The usage of the app requires the user to consent to this, as there is otherwise no point in us offering the service which could otherwise incur cloud computing costs.
    \item The application does not capture any data defined by the GDPR as "personal", except for the user's email. Evaluation metrics, while potentially changing over the course of the project, should not allow the de-anonymisation of a given user.
    \item Data will be sent and stored using secure channels, using third-party GDPR compliant vendors where applicable.
    \item In the unlikely event of a data leak, users will be immediately informed via email what was leaked. However, none of this data should be particularly harmful to a user even if this should happen.
    \item Users may at any time erase all their usage data within the settings menu.
\end{itemize}